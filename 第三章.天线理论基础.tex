\chapter{天线理论基础}

运动电子, Larmor公式.

短(尺度远小于波长)偶极子天线. (Gauss单位制)
\begin{equation}
    E=\frac{q\dot{v}\sin\theta}{rc^2},
\end{equation}
全天线
\begin{equation}
    E=\int_{-l/2}^{+l/2}\frac{\d q}{\d z}\frac{\dot{v}\sin\theta}{rc^2}\d z,
\end{equation}
$\dot{v}=-i\omega v$,
\begin{equation}
    E=\frac{-i\omega\sin\theta}{rc^2}\int_{-l/2}^{+l/2}I\d z,
\end{equation}
假设$I=I_0e^{-i\omega t}\left[1-\frac{\left\lvert z\right\rvert }{l/2}\right]$,
\begin{equation}
    E=\frac{-i\omega\sin\theta}{rc^2}\frac{I_0l}{2}e^{-i\omega t}=\frac{-i\pi\sin\theta}{c}\frac{I_0l}{\lambda}\frac{e^{-i\omega t}}{r},
\end{equation}
\begin{equation}
    S=\frac{c}{4\pi}E^2=\frac{c}{4\pi}\left(\frac{I_0l}{\lambda}\frac{\pi}{c}\right)^2\frac{\sin^2\theta}{r^2}\cos^2(\omega t+\frac{\pi}{2}),
\end{equation}
\begin{equation}
    \left\langle S\right\rangle=\frac{1}{2}\frac{c}{4\pi}\left(\frac{I_0l}{\lambda}\frac{\pi}{c}\right)^2\frac{\sin^2\theta}{r^2}\propto\sin^2\theta.
\end{equation}
实际一般$l\approx\lambda/2$, $I=I_0e^{-i\omega t}\cos(2\pi z/\lambda)$.

辐射电阻$R$,
\begin{equation}
    \left\langle P\right\rangle := \left\langle I^2\right\rangle R=\frac{1}{2}I_0^2R.
\end{equation}

功率增益$G(\theta, \phi)$,
\begin{equation}
    G(\theta, \phi):=\frac{P(\theta, \phi)}{\left\langle P(\theta, \phi)\right\rangle }.
\end{equation}
$G(\text{dB})=10\log_{10}(G)$.

有效接受面积$A_\text{e}$,
\begin{equation}
    A_\text{e}=\frac{P_\nu}{S_\text{matched}}.
\end{equation}
$\left\langle A_\text{e}\right\rangle =\frac{\lambda^2}{4\pi}$, 短波无方向性则效率低.
\begin{equation}
    A_\text{e}(\theta, \phi)=\frac{\lambda^2G(\theta, \phi)}{4\pi}.
\end{equation}

天线温度$T_\text{A}$,
\begin{equation}
    T_\text{A}:=\frac{P_\nu}{k}.
\end{equation}
