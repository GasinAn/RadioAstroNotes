% 编译方式: xelatex*2
\documentclass{ctexbook}
\usepackage{amsfonts}
\usepackage{amsmath}
\usepackage{amssymb}
\usepackage{hyperref}
\usepackage{syntonly}
\usepackage{IEEEtrantools}
%\syntaxonly
\pagestyle{plain}
\makeatletter
\newcommand{\starttoc}{
    \chapter*{\contentsname}
    \@starttoc{toc}
}
\makeatother
%\renewcommand{\tableofcontents}{\twocolumn\starttoc\onecolumn}
\hypersetup{
	colorlinks,
	linkcolor=blue,
	filecolor=pink,
	urlcolor=cyan,
	citecolor=red,
}
\def\b{\boldsymbol}
\def\d{\mathrm{d}}
\title{射电天文学笔记}
\author{GasinAn}
\begin{document}
    \maketitle
    \noindent Copyright \textcopyright~2022 by GasinAn

\ 

\noindent All rights reserved. No part of this book may be reproduced, 
in any form or by any means, without permission in writing from the publisher, except by a BNUer.

\ 

\noindent The author and publisher of this book have used their best efforts
in preparing this book. These efforts include the development, research, and testing of the theories,
technologies and programs to determine their effectiveness.
The author and publisher make no warranty of any kind, express or implied,
with regard to these techniques or programs contained in this book.
The author and publisher shall not be liable in any event of incidental or consequential damages
in connection with, or arising out of, the furnishing, performance, or use of these techniques or programs.

\ 

\noindent Printed in China

    \tableofcontents
    \chapter{射电信号的探测方法}

射电(10MHz--1THz) $\ne$无线电(3kHz--3GHz).

白天可测: 波长远大大气尘埃, 无散射; 太阳射电信号少.

排除水汽: 高地, 旱地.

\begin{itemize}
    \item 镜面精度低($\lambda/20$).
    \item $\lambda$大, 在$\lambda^3$范围内放很多带点粒子, 形成相干辐射.
    \item 波长远大星际尘埃, 透明.
    \item $h\nu/kT\ll1$, 所有天体都辐射.
    \item $\theta\sim\lambda/D$, 需要大$D$.
\end{itemize}

    \chapter[射电辐射基础]{射电辐射基础\footnote{梦回天体物理导论.}}

$I_\lambda$: 垂直于单位面积方向的单位立体角内单位时间通过的单位波长的能量. 
$F_\lambda$: 单位面积的所有立体角内单位时间通过的单位波长的能量.
\begin{equation}
    I_\lambda=\frac{\d E}{\cos\theta\,\d\sigma\,\d\Omega\,\d t\,\d\lambda}.
\end{equation}
\begin{equation}
    F_\lambda = \int I_\lambda \cos\theta \,\mathrm{d}\Omega.
\end{equation}

$1\,\text{Jy}=10^{-26}\,\text{W/(m}^2\!\cdot\text{Hz)}$. $1\,\text{erg}=10^{-7}\,\text{J}$.

\begin{equation}
    \frac{\d I_\nu}{I_\nu}=-\kappa_\nu\,\d s.
\end{equation}
\begin{equation}
    \tau_\nu=\int_{s_\text{in}}^{s_\text{out}}\kappa_\nu\,\d s.
\end{equation}
\begin{equation}
    I_\nu(s_\text{out})=I_\nu(s_\text{in})e^{-\tau_\nu}.
\end{equation}

\begin{equation}
    \d I_\nu=j_\nu\,\d s.
\end{equation}
\begin{equation}
    \frac{\d I_\nu}{\d s}=-\kappa_\nu I_\nu+j_\nu.
\end{equation}

低频$B_\nu\approx\frac{2kT\nu^2}{c^2}$, 亮温度
\begin{equation}
    T_\text{b}:=\frac{I_\nu c^2}{2k\nu^2}.
\end{equation}

    \chapter{天线理论基础}

运动电子, Larmor公式.

短(尺度远小于波长)偶极子天线.
\begin{equation}
    E=\frac{q\dot{v}\sin\theta}{rc^2},
\end{equation}
全天线
\begin{equation}
    E=\int_{-l/2}^{+l/2}\frac{\d q}{\d z}\frac{\dot{v}\sin\theta}{rc^2}\d z,
\end{equation}
$\dot{v}=-i\omega v$,
\begin{equation}
    E=\frac{-i\omega\sin\theta}{rc^2}\int_{-l/2}^{+l/2}I\d z,
\end{equation}
假设$I=I_0e^{-i\omega t}\left[1-\frac{\left\lvert z\right\rvert }{l/2}\right]$,
\begin{equation}
    E=\frac{-i\omega\sin\theta}{rc^2}\frac{I_0l}{2}e^{-i\omega t}=\frac{-i\pi\sin\theta}{c}\frac{I_0l}{\lambda}\frac{e^{-i\omega t}}{r},
\end{equation}
\begin{equation}
    S=\frac{c}{4\pi}E^2=\frac{c}{4\pi}\left(\frac{I_0l}{\lambda}\frac{\pi}{c}\right)^2\frac{\sin^2\theta}{r^2}\cos^2(\omega t+\frac{\pi}{2}),
\end{equation}
\begin{equation}
    \left\langle S\right\rangle=\frac{1}{2}\frac{c}{4\pi}\left(\frac{I_0l}{\lambda}\frac{\pi}{c}\right)^2\frac{\sin^2\theta}{r^2}\propto\sin^2\theta.
\end{equation}
实际一般$l\approx\lambda/2$, $I=I_0e^{-i\omega t}\cos(2\pi z/\lambda)$.

辐射电阻$R$,
\begin{equation}
    \left\langle P\right\rangle := \left\langle I^2\right\rangle R=\frac{1}{2}I_0^2R.
\end{equation}

功率增益$G(\theta, \phi)$,
\begin{equation}
    G(\theta, \phi):=\frac{P(\theta, \phi)}{\left\langle P(\theta, \phi)\right\rangle }.
\end{equation}
$G(\text{dB})=10\log_{10}(G)$.

有效接受面积$A_\text{e}$,
\begin{equation}
    A_\text{e}=\frac{P_\nu}{S_\text{matched}}.
\end{equation}
$\left\langle A_\text{e}\right\rangle =\frac{\lambda^2}{4\pi}$, 短波无方向性效率低.
\begin{equation}
    A_\text{e}(\theta, \phi)=\frac{\lambda^2G(\theta, \phi)}{4\pi}.
\end{equation}

天线温度$T_\text{A}$,
\begin{equation}
    T_\text{A}:=\frac{P_\nu}{k}.
\end{equation}

    \chapter{反射天线}

一维反射面, 频率$\omega$, 强度$g(x)$照射面, 反射到远处, $l:=\sin\theta$, 远处强度$f(l)$, $u:=x/\lambda$, 刚好Fourier变换
\begin{equation}
    f(l)=\int g(u)e^{-2\pi ilu}\d u.
\end{equation}

    \chapter{干涉天线}

窄带, 假设频率都为$\omega$, 距离$b$, 源方向$\hat{s}$, $\vec{b}$和$\hat{s}$夹角$\theta$, 两天线电压$V_1$, $V_2$, 时间延迟$\tau_\text{g}$,
\begin{equation}
    V_1V_2=\frac{V^2}{2}\left[\cos(2\omega t+\omega\tau_\text{g})+\cos(\omega\tau_\text{g})\right],
\end{equation}
\begin{equation}
    R:=\left\langle V_1V_2\right\rangle =\frac{V^2}{2}\cos(\omega\tau_\text{g}).
\end{equation}
$\phi:=\omega\tau_\text{g}=\frac{b\omega}{c}\cos\theta$, $\Delta\phi=2\pi\to\Delta\theta=\lambda/(b\sin\theta)$, 若$b\sin\theta\gg\lambda$, 则小$\Delta\theta$得$\Delta\phi$.

非点源,
\begin{equation}
    R_\text{c}=\int I_\omega(\hat{s})\cos(\omega\tau_\text{g})\,\d\Omega.
\end{equation}
$I$分解为奇偶宇称, $\cos$偶宇称, 奇宇称人为加$\frac{\pi}{2}$相位,
\begin{equation}
    R_\text{s}=\int I_\omega(\hat{s})\sin(\omega\tau_\text{g})\,\d\Omega.
\end{equation}
定义
\begin{equation}
    R=\int I_\omega(\hat{s})\exp(-i\omega\tau_\text{g})\,\d\Omega.
\end{equation}

\end{document}
