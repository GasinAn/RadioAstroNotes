\chapter{干涉天线}

窄带, 假设频率都为$\omega$, 距离$b$, 源方向$\hat{s}$, $\vec{b}$和$\hat{s}$夹角$\theta$, 两天线电压$V_1$, $V_2$, 时间延迟$\tau_\text{g}$,
\begin{equation}
    V_1V_2=\frac{V^2}{2}\left[\cos(2\omega t+\omega\tau_\text{g})+\cos(\omega\tau_\text{g})\right],
\end{equation}
\begin{equation}
    R:=\left\langle V_1V_2\right\rangle =\frac{V^2}{2}\cos(\omega\tau_\text{g}).
\end{equation}
$\phi:=\omega\tau_\text{g}=\frac{b\omega}{c}\cos\theta$, $\Delta\phi=2\pi\to\Delta\theta=\lambda/(b\sin\theta)$, 若$b\sin\theta\gg\lambda$, 则小$\Delta\theta$得$\Delta\phi$.

\begin{equation}
    R=\frac{V_1(\theta)V_2(\theta)}{2}\cos\left(\frac{b\omega}{c}\cos\theta\right)=\frac{V_1(\theta)V_2(\theta)}{2}\cos\left(\frac{b\cos\theta}{\lambda}\right).
\end{equation}
$\frac{V_1(\theta)V_2(\theta)}{2}$一般Gauss. 多天线, 不同$b$叠加得Gauss.

非点源,
\begin{equation}
    R_\text{c}=\int I_\omega(\hat{s})\cos(\omega\tau_\text{g})\,\d\Omega.
\end{equation}
$I$分解为奇偶宇称, $\cos$偶宇称, 奇宇称人为加$\frac{\pi}{2}$相位,
\begin{equation}
    R_\text{s}=\int I_\omega(\hat{s})\sin(\omega\tau_\text{g})\,\d\Omega.
\end{equation}
定义
\begin{equation}
    R=\int I_\omega(\hat{s})\exp(-i\omega\tau_\text{g})\,\d\Omega.
\end{equation}
